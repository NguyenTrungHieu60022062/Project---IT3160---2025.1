\documentclass[12pt]{article}
\usepackage[utf8]{inputenc}
\usepackage[vietnamese]{babel}
\usepackage{graphicx}
\usepackage{amsmath}
\usepackage{amssymb}

\begin{document}
\begin{center}
\vspace*{-4cm}
\textbf{\normalsize ĐẠI HỌC BÁCH KHOA HÀ NỘI}\\
\textbf{\normalsize TRƯỜNG CÔNG NGHỆ THÔNG TIN VÀ TRUYỀN THÔNG}
\rule{0.5\textwidth}{0.4pt}\\
\vspace*{3cm}
\textbf{\Huge BÀI TẬP LỚN}\\
\vspace*{0.5cm}
\textbf{\Large MÔN: NHẬP MÔN TRÍ TUỆ NHÂN TẠO}\\
\vspace*{3cm}
\textbf{\Large ĐỀ TÀI: PHẦN MỀM TÌM ĐƯỜNG ĐI KHU VỰC PHƯỜNG CẦU GIẤY, HÀ NỘI}\\
\vspace*{2cm}
\textbf{\large NHÓM 24}\\
\vspace*{0.5cm}
\textbf{\normalsize Bạn 1}\\
\textbf{\normalsize Bạn 2}\\
\textbf{\normalsize Bạn 3}\\
\textbf{\normalsize Bạn 4}\\
\textbf{\normalsize Bạn 5}\\
\vspace*{2cm}
\text{\normalsize MÃ LỚP HỌC: 162278}\\
\text{\normalsize GIÁO VIÊN HƯỚNG DẪN: PGS.TS TRẦN ĐÌNH KHANG}\\
\end{center}

\newpage

\tableofcontents

\newpage
\section{Mô tả bài toán}
\begin{itemize}
    \item Xét một mảnh bản đồ của phường Cầu Giấy, Hà Nội. Bài toán yêu cầu tìm đường đi giữa 2 điểm A và B bất kỳ và biểu diễn đường đi đó trên bản đồ.
    \item Ràng buộc: 
    \begin{itemize}
    \item Chỉ được phép đi trên các tuyến đường, không được phép đi xuyên qua các công trình.
    \item ... (thời tiết)
    \item ... (lưu lượng)
    \item ...
    \end{itemize}
\end{itemize}
\end{document}
